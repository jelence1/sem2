\chapter{Uvod}

Internet stvari (engl. \textit{Internet of things - IoT}) nova je paradigma koja je tradicionalni način života promijenila u stil života visoke tehnologije. U sklopu IoT pokreta razvio se koncept sveprisutnog računarstva koji označava kompletnu prostornu i vremensku prisutnost pametnih uređaja u svakodnevnom životu. Prema principima koncepta \cite{sverac}, primarna svrha računala je pružiti pomoć na intuitivan način. 

Sveprisutni sustavi, kao posljedica okoline i zahtjeva, razlikuju se po svojim svojstvima od klasičnih računalnih sustava. Neka od njih su autonomnost, konkurentnost, svjesnost konteksta te računanje u stvarnom vremenu (engl. \textit{real-time computation}). Sustav koji zadovoljava vremenska ograničenja na odziv mora imati hardver koji podržava rad u stvarnom vremenu, no istovremeno pruža bežičnu povezivost kao dio sveprisutnog sustava. Jedan od takvih uređaja je modul ESP32-C3 tvrtke \textit{Espressif}, koji osim rada u stvarnom vremenu podržava i bežičnu povezivost putem Bluetootha i Wi-Fi veze. Za izradu ovog rada odabran je razvojni sustav ESP32-C3-DevKitM-1. Isto tako, za pohranu podataka u takvim sustavima potrebna je i baza podataka koja podržava rad u stvarnom vremenu. Baze podataka u stvarnom vremenu trenutno osvježavaju podatke nakon promjene, što omogućava da više uređaja spojenih na bazu dohvaćaju i sinkroniziraju podatke u stvarnom vremenu. Baza podataka koja pruža takvu mogućnost je Firebase, baza podataka u oblaku tvrtke \textit{Google}.

Ovaj seminar analizira mogućnosti koje pruža ESP32-C3-DevKitM-1 u razvoju aplikacija koristeći Wi-Fi te kako povezati modul s bazom podataka Firebase. Opisana su programska aplikacijska sučelja (engl. \textit{Application Programming Interface - API}) koje modul podržava za Wi-Fi povezivanje i demo aplikacije uz pripadna sučelja. Dan je osnovni pregled funkcionalnosti koje nudi baza podataka Firebase te su navedene njezine prednosti i ograničenja. Isto tako, opisan je postupak povezivanja modula i baze podataka, koje je popraćeno jednostavnom demo aplikacijom na modulu. Također, napravljena je mobilna aplikacija za simulaciju stvarnog IoT uređaja koji koristi modul i bazu podataka.

Rad je podijeljen u cjeline kako slijedi. Drugo poglavlje „\textit{Razvojni sustav ESP32-C3-DevKitM-1}“ opisuje osnovne karakteristike korištenog razvojnog sustava kao ciljane hardverske platforme te su opisane najvažnije značajke W-Fi tehnologije, kao i API-ji koji se mogu koristiti uz razvojni sustav. U trećem poglavlju „\textit{Baza podataka Firebase}“ dan je pregled baze podataka i njezinih glavnih značajki te je opisan postupak povezivanja (programsko spajanje?) razvojnog sustava s bazom podataka. U četvrtom poglavlju „\textit{Modeliranje stvarnog IoT sustava}“ opisan je primjer primjene sustava koji koristi opisane tehnologije uz mobilnu aplikaciju.

\eject